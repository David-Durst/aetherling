\documentclass[11pt,fleqn]{article}
\usepackage{amsmath}
\usepackage{changepage}% http://ctan.org/pkg/changepage
\numberwithin{equation}{subsection}
\usepackage[parfill]{parskip}
\setlength{\marginparwidth}{0pt}
\usepackage{layout}
\begin{document}
\title{Aetherling Nodes and Example Pipelines}

\section{Notation}
Single cycle, combinational logic has the form: 
\begin{adjustwidth}{2.5 em}{0 pt}
    input 1 $\rightarrow$ ... $\rightarrow$ input n $\rightarrow$ output
\end{adjustwidth}

Each input or output has type T or T[p]

T is a base type, it can contain nested types like arrays that are not relevant
for the current operation.

T[p] is an array of length p of T's. T[p][q] is an array of length p of arrays of length q of T's.

Multiple cycle, sequential logic has the form:
\
\{c, T\} is a stream of T's. There is one T per clock cycle for c cycles.

\section{Basic Operations}

\subsection{Combinational Operations}
These operations require only combinational logic. Therefore, they do not
require a clock nor ready-valid signals. All their input and output types are over
an implicit stream. This means that that they operate on one input every clock
cycle and emit one output every clock cycle.

\[
    \text{tuple} :: S \rightarrow T \rightarrow (S,T)
\]
\[
    \text{lb}\;p\;w :: T[p] \rightarrow T[w+p-1] 
\]
\[
    \text{overlap\_partition}\;p\;w :: T[w+p-1] \rightarrow T[p][w]
\]
\[
    \text{partition}\;p\;k :: T[k] \rightarrow T[k/p][p]
\]
\[
    \text{flatten}\;p\;k :: T[k/p][p] \rightarrow T[k]
\]
\[
    \text{map}\;p\;f :: S[p] \rightarrow T[p]   \qquad \text{s.t.}\;f :: S \rightarrow T
\]
\[
    \text{reduce}\;p\;f :: T[p] \rightarrow T \qquad \text{s.t.}\;f :: (T,T)\rightarrow T
\]
\[
    \text{up}\;k :: T \rightarrow T[k]
\]
\[
    \text{down}\;k :: T[k] \rightarrow T 
\]
\[
    \text{zip} :: (S[k],T[k]) \rightarrow (S,T)[k]
\]
\[
    \text{unzip} :: (S,T)[k] \rightarrow (S[k], T[k])
\]
\[
    \text{mem\_read}\;p :: () \rightarrow T[p] 
\]
\[
    \text{mem\_write}\;p :: T[p] \rightarrow ()
\]

\subsection{Sequential Operations}
These operations require sequential logic. Therefore, they require a clock and
ready-valid signals. Their input and output types are over explicit streams.

\[
    \text{serialize}\;k :: \{1, T[k]\} \rightarrow \{k, T\}
\]
\[
    \text{deserialize}\;k :: \{k, T\} \rightarrow \{1, T[k]\}
\]

\subsubsection{Ready-Valid Connector Operations}
Any sequential block of


\section{Basic Applications}
These are simple combinations of the 
\subsection{Passthrough}
\begin{enumerate}
    \item mem\_write 1 \$ mem\_read 1 
    \item mem\_write t \$ mem\_write t
\end{enumerate}

\subsection{Array-Stream Conversions}
\begin{enumerate}
    \item mem\_write 1 \$ deserialize t \$ mem\_read t
        \subitem Note that mem\_read fires onces every t'th clock cycle
    \item mem\_write 1 \$ deserialize t \$ serialize t \$ mem\_read 1
        \subitem Note mem\_write fires every t'th clock cycle
\end{enumerate}

\subsection{Map}
\begin{enumerate}
    \item mem\_write 1 \$ map 1 f \$ mem\_read 1
    \item mem\_write t \$ map t f \$ mem\_read t
    \item mem\_write t \$ map t f2 \$ map t f1 \$ mem\_read t
\end{enumerate}

\subsection{Reduce}
\begin{enumerate}
    \item mem\_write 1 \$ reduce t f \$ mem\_read t
    \item mem\_write 1 \$ reduce t f \$ deserialize t f \$ mem\_read 1
        \subitem Note everything after deserialize fires every t'th clock cycle
\end{enumerate}

\subsection{Array Dimension Conversions}
\begin{enumerate}
    \item mem\_write $\frac{t}{p}$ \$ partition p t \$ mem\_read t
        \subitem Note that the element type mem\_write is writing is T[p],
        and it writes $\frac{t}{p}$ of them every clock.
    \item mem\_write t \$ flatten t \$ partition t \$ mem\_read t
    \item mem\_write 1 \$ flatten t \$ partition t \$ deserialize t \$ mem\_read t
        \subitem Note everything after deserialize fires onces every t'th clock 
        cycle
    \item mem\_write 1 \$ down t \$ up t \$ mem\_read 1
\end{enumerate}

\section{Advanced Applications}

\subsection{}
These are operations that are combinations of the basic ones. They are also
applica
\begin{enumerate}
    \item t
\end{enumerate}


\end{document}
