\documentclass[11pt,fleqn]{article}
\usepackage{amsmath}
\usepackage{changepage}% http://ctan.org/pkg/changepage
\numberwithin{equation}{subsection}
\usepackage[parfill]{parskip}
\setlength{\marginparwidth}{0pt}
\usepackage{layout}
\begin{document}
\title{Aetherling Nodes and Example Pipelines}

\section{TODOS}
\begin{enumerate}
    \item Need to add area/time estimates for each element and application
    \item How to keep input streams to map in sequence?
    \item How does this work if I have a pipe operator that makes zipn more useful?
\end{enumerate}

\section{Type Notation}

\subsection{Combinational Element Types}
A combinational element is one that is implemented using only combinational logic.
The type signature of combinational elements is:
\begin{adjustwidth}{2.5 em}{0 pt}
    input 0 type $\rightarrow$ ... $\rightarrow$ input n-1 type $\rightarrow$ output type
\end{adjustwidth}
Since these combinational elements process all inputs and output every 
clock cycle, they do not consume nor produce ready or valid signals. If they
are contained within a larger, nested pipeline that has ready-valid signals,
they do no affect those signals. These elements are always ready and produce 
valid data on every cycle that they receive valid data.

Each input or output has type T or T[p]. T is a base type, and it can contain 
nested types like arrays that are not relevant for the current operation. T[p] 
is an array of length p of T's. T[p][q] is an array of length p of arrays of 
length q of T's.

\subsection{Sequential Element Types}
A sequential element is one that has sequential logic in its implementation.
The type signature of sequential elements is:
\begin{adjustwidth}{2.5 em}{0 pt}
    \{input 0 num cycles, input 0 type per cycle\} $\rightarrow$ ...
    $\rightarrow$ \\
    \{input n-1 num cycles, input n-1 type per cycle\} $\rightarrow$ \\ 
    \{ouput num cycles, output type per cycle\}
\end{adjustwidth}

One of the inputs or outputs of a stream may have different types during different
clock cycles in a stream. For example, a sequential reduce may emit invalid 
data for most of the stream's cycles and then emit the result on the final
cycle of the stream. Instead of \{cycles, type for all cycles\}, 
this type is represented in the following way:
\begin{adjustwidth}{2.5 em}{0 pt}
    \{type at cycle 0, type at cycle 1, ..., type at cycle n-1\}
\end{adjustwidth}
For short-hand where a type is the same for multiple cycles:
\begin{adjustwidth}{2.5 em}{0 pt}
    \{type at cycle 0(0:n-2), type at cycle n-1\}
\end{adjustwidth}
The invalid type is represented as $\emptyset$.

These elements interfaces also must have clock inputs and may have ready-valid 
inputs and outputs. These ready-valid handshake ports indicate:
\begin{enumerate}
        \item ready
            \subitem input: indicates to this sequential element that the next 
            one in the pipeline has completed its prior input stream and is 
            ready to receive more input.
            \subitem output: indicates to the previous sequential element in the 
            pipeline that this one has completed its prior input stream and is
            ready to receive more input.
        \item valid
            \subitem input: indicates to this sequential element that the previous
            one in the pipeline is emitting valid data that this sequential 
            element can use as input.
            \subitem output: indicates to the next sequential element in the
            pipeline that this one is emitting valid data that the next one
            can use as input.
\end{enumerate}

\section{Elements}

\subsection{Basic Combinational Elements}
These are elements that are built using logical primitives not available in
Aetherling elements.

\begin{enumerate}
    \item tuple :: S $\rightarrow$ T $\rightarrow$ (S,T)
    \item lb p w :: T[p] $\rightarrow$ T[w+p-1]
    \item overlap\_partition p w :: T[w+p-1] $\rightarrow$ T[p][w]
    \item partition p k :: T[k] $\rightarrow$ T[k/p][p]
    \item flatten p k :: T[k/p][p] $\rightarrow$ T[k]
    \item map p f :: $S_0$[p] $\rightarrow$ ... $\rightarrow$ $S_{m-1}$[p] $\rightarrow$ 
        T[p]
        \subitem s.t. f :: $S_0 \rightarrow$ ... $\rightarrow S_{m-1} \rightarrow$ T 
    \item reduce p f :: T[p] $\rightarrow$ T
        \subitem s.t. f :: (T,T)$\rightarrow$ T
    \item up k :: T $\rightarrow$ T[k]
    \item down k :: T[k] $\rightarrow$ T
    \item zip :: (S[k],T[k]) $\rightarrow$ (S,T)[k]
    \item unzip :: (S,T)[k] $\rightarrow$ (S[k], T[k])
    \item mem\_read p :: $\emptyset$ $\rightarrow$ T[p]
    \item mem\_write p :: T[p] $\rightarrow$ $\emptyset$
\end{enumerate}

\subsection{Sequential Elements}

\subsubsection{Rate Changing Elements}

\begin{enumerate}
    \item serialize p :: \{T[p], $\emptyset$(1:p-1)\} $\rightarrow$ \{p, T\}
    \item deserialize p :: \{p, T\} $\rightarrow$ \{$\emptyset$(0:p-2), T[p]\}
\end{enumerate}

\subsubsection{Basic Sequential Versions Of Basic Combinational Elements}
These are the sequential elements that cannot be implemented using only the basic
combinational elements. In an ideal world, there would only be basic
combinational logic blocks and rate changing sequential elements, but these
elements are better implemented using logical blocks that do not belong in
Aetherling.

\begin{enumerate}
    \item reduce\_seq k f :: \{T[k], $\emptyset$(1:k-1)\} $\rightarrow$ \{$\emptyset$(0:k-2), T\}
        \subitem s.t. f :: (T,T)$\rightarrow$ T
    \item reduce\_seq\_stream k f :: \{k, T\} $\rightarrow$ \{$\emptyset$(0:k-2), T\}
        \subitem s.t. f :: (T,T)$\rightarrow$ T
    \item up\_seq k :: \{T, $\emptyset$(1:k-1)\} $\rightarrow$ \{k, T\}
    \item down\_seq k :: \{k, T\} $\rightarrow$ \{T, $\emptyset$(1:k-1)\}
\end{enumerate}

\subsubsection{Composed Sequential Versions Of Basic Combinational Elements}
These are the sequential elements that can be implemented using only the basic
combinational elements.

\begin{enumerate}
    \item map\_seq k f :: \{$S_0$[k], $\emptyset$(1:p-1)\} $\rightarrow$ ...
        $\rightarrow$ \{$S_{m-1}$[p], $\emptyset$(1:p-1)\} $\rightarrow$
        \{$\emptyset$(0:p-2), T[p]\}
        \subitem s.t. f :: $S_0 \rightarrow$ ... $\rightarrow S_{m-1} \rightarrow$ T
        \subitem This map takes all the inputs in on the first cycle of the 
        stream and emits all the outputs on the final cycle of the stream.
        \subitem implementation: map\_seq p f = deserialize p \$ f \$ \\ map m (serialize p)
        \subitem note that in the above implementation, the type for serialize 
        contains all the different input types to map
    \item map\_seq\_stream p f :: \{$S_0$[p], $\emptyset$(1:p-1)\} $\rightarrow$ ...
        $\rightarrow$ \{$S_{m-1}$[p], $\emptyset$(1:p-1)\} $\rightarrow$
        \{p, T\}
        \subitem s.t. f :: $S_0 \rightarrow$ ... $\rightarrow S_{m-1} \rightarrow$ T
        \subitem This map takes all the inputs in on the first cycle of the 
        stream and emits one output on each cycle of the stream.
        \subitem implementation: map\_seq\_stream p f = f \$ map m (serialize p)
    \item map\_partially\_par k p f :: \{$S_0$[k], $\emptyset$(1:$\frac{k}{p}-1$)\} 
        $\rightarrow$ ... $\rightarrow$ 
        \{$S_{m-1}$[k], $\emptyset$(1:$\frac{k}{p}-1$)\} $\rightarrow$
        \{$\emptyset$(0:$\frac{k}{p}-2$), T[p]\}
        \subitem s.t. f :: $S_0 \rightarrow$ ... $\rightarrow S_{m-1} \rightarrow$ T
        \subitem implementation: map\_partially\_par k p f = flatten p k \$ \\ 
        deserialize $\frac{k}{p}$ \$ map p f \$ map m (serialize $\frac{k}{p}$) \$ map m (partition p k)
    \item map\_partially\_par\_stream k p f :: \{$S_0$[k], $\emptyset$(1:$\frac{k}{p}-1$)\} 
        $\rightarrow$ ... $\rightarrow$ 
        \{$S_{m-1}$[k], $\emptyset$(1:$\frac{k}{p}-1$)\} $\rightarrow$
        \{$\frac{k}{p}$, T[p]\}
        \subitem s.t. f :: $S_0 \rightarrow$ ... $\rightarrow S_{m-1} \rightarrow$ T
        \subitem implementation: map\_partially\_par k p f = map p f \$ \\ 
        map m (serialize $\frac{k}{p}$) \$ map m (partition p k)
    \item reduce\_partially\_par k p f :: \{T[p], $\emptyset$(1:$\frac{k}{p}-1$)\} 
        $\rightarrow$ \{$\emptyset$(0:$\frac{k}{p}-2$), T\}
        \subitem s.t. f :: (T,T) $\rightarrow$ T
        \subitem implementation: reduce\_seq k/p f \$ map\_seq k/p (reduce p f) 
        \$ partition p
    \item reduce\_partially\_par\_stream k p f :: \{$\frac{k}{p}$, T[p]\}
        $\rightarrow$ \{$\emptyset$(0:$\frac{k}{p}-2$), T\}
        \subitem s.t. f :: (T,T) $\rightarrow$ T
        \subitem implementation: reduce\_seq\_stream k/p f \$ reduce p f
\end{enumerate}

\section{Basic Applications}
These are simple combinations of the basic elements.

\subsection{Passthrough}
\begin{enumerate}
    \item mem\_write 1 \$ mem\_read 1
    \item mem\_write t \$ mem\_read t 
\end{enumerate}

\subsection{Array-Stream Conversions}
\begin{enumerate}
    \item mem\_write 1 \$ deserialize t \$ mem\_read t
        \subitem Note that mem\_read fires onces every t'th clock cycle
    \item mem\_write t \$ deserialize t \$ serialize t \$ mem\_read t
        \subitem Note mem\_write and mem\_read fires every t'th clock cycle
    \item mem\_write 1 \$ serialize t \$ deserialize t \$ mem\_read 1
        \subitem Note this has a t cycle startup before mem\_write starts writing
\end{enumerate}

\subsection{Map}
\begin{enumerate}
    \item mem\_write 1 \$ map 1 (+1) \$ mem\_read 1
    \item mem\_write t \$ map t (+1) \$ mem\_read t
    \item mem\_write t \$ map t (+1) \$ map t f1 \$ mem\_read t
    \item mem\_write t \$ map\_seq t (+1) \$ mem\_read t
        \subitem Note mem\_write and mem\_read fire every t'th clock cycle
    \item mem\_write 1 \$ map\_seq\_stream t (+1) \$ mem\_read t
        \subitem Note mem\_read fires every t'th clock cycle
    \item mem\_write t \$ map\_partially\_par t p (+1) \$ mem\_read t
        \subitem Note mem\_write and mem\_read fire every $\frac{t}{p}$'th clock cycle
    \item mem\_write 1 \$ map\_partially\_par\_stream t p (+1) \$ mem\_read t
        \subitem Note mem\_read fires every $\frac{t}{p}$'th clock cycle
\end{enumerate}

\subsection{Reduce}
\begin{enumerate}
    \item mem\_write 1 \$ reduce t (+) \$ mem\_read t
    \item mem\_write 1 \$ reduce t (+) \$ deserialize t f \$ mem\_read 1 
        \subitem Note everything after deserialize fires every t'th clock cycle
    \item mem\_write 1 \$ reduce\_seq t (+) \$ mem\_read t
        \subitem Note mem\_write and mem\_read fire every t'th clock cycle
    \item mem\_write 1 \$ reduce\_seq\_stream t (+) \$ mem\_read 1
    \item mem\_write 1 \$ reduce\_partially\_par t p (+) \$ mem\_read t
        \subitem Note mem\_read fires every t'th clock cycle
    \item mem\_write 1 \$ reduce\_partially\_par\_stream t p (+) \$ mem\_read p
        \subitem Note mem\_write fires every $\frac{t}{p}$'th clock cycle
\end{enumerate}

\subsection{Array Dimension Conversions}
\begin{enumerate}
    \item mem\_write $\frac{t}{p}$ \$ partition p t \$ mem\_read t
        \subitem Note that the element type mem\_write is writing is T[p], 
        and it writes $\frac{t}{p}$ of them every clock.
    \item mem\_write t \$ flatten t \$ partition t \$ mem\_read t 
    \item mem\_write 1 \$ flatten t \$ partition t \$ deserialize t \$ mem\_read t
        \subitem Note everything after deserialize fires onces every t'th 
        clock cycle
    \item mem\_write 1 \$ down t \$ up t \$ mem\_read 1
    \item mem\_write 1 \$ down\_seq t \$ up\_seq t \$ mem\_read 1
\end{enumerate}

\end{document}
